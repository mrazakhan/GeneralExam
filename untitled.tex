\textit{Version 1. 10/25/2015}   
The focus of my research work up till now has been on Computational Social Science which can be considered as an intersection of Social Sciences and Computer Science. Computational Social Science is a big field employing different algorithmic tools and techniques to solve different problems in different areas of social sciences. My focus within Computational Social Science is on Human Behavior Analysis using Data Science Techniques. Study in this area will enable me to see that how different aspects of Human Behavior can be analyzed better using computational and statistical techniques.\\ 
Application of big data tools and techniques is becoming more and more important every day as the magnitude of information and data available for analysis has grown tremendously in the last decade. As a result, many research projects have been carried out which employ new and efficient techniques for mining human data. My research is focused on application of these big data tools and techniques on social problems. Some of the problems that I have been working on include Customer Churn Prediction using Telecommunication CDR data, Migration Trends Analysis using Telecommunication CDR Data and Latent Profile Features Predication using Twitter Data.\\ 

Human Behavior Analysis using Data Science is a natural progression of my background in Computer Science and theme of the projects that I have been involved in as a PhD Student in the Information School during the last 2 years.   Reading list for my general exam will consist of research material involving major techniques in data science and the application of the these techniques on the problems related to Social Science.   
\section{Methods (Data Science (Machine Learning) Literature)}
\begin{enumerate}   
\item  Tomaso Poggio and Steve Smale (2003).  The Mathematics of Learning: Dealing with Data\cite{Poggio_2005} 
\item  Z. Ghahramani (2004).  Unsupervised Learning \cite{Ghahramani_2004}
\item  X. Wu et al. (2008). Top 10 algorithms in data mining \cite{2009}  
\item  Hastie et al. (2009). Elements of Statistical Learning \cite{StatisticalLearning_2009}  
\item  Esther Duflo et al. (2006). Using Randomization in Development Economics Research: A Toolkit \cite{Duflo}   
\item  Rajaraman et al. (2009). Mining of Massive Datasets. \cite{Rajaraman_2009}  
\item  Yaser S. AbuMostafa et al. (2012). Learning from Data \cite{Abu-Mostafa:2012:LD:2207825}  
\end{enumerate}  

\subsection{Data Science and Development Economics}  \\  
\begin{enumerate}  
\item  Esther Duflo (2000).  Schooling and Labor Market Consequences of School Construction in Indonesia: Evidence from an Unusual Policy Experiment \cite{Duflo_2000} 
\item Daniel Bjorkegren (2015).  The Adoption of Network Goods \cite{Bjorkegren} 
\end{enumerate}  
\\  
\subsection{Data Science and Measurements}  
\begin{enumerate}  
\item  P. Deville et al. (2014).  Dynamic Population Mapping using Mobile Phone Data\cite{Deville_2014} 
\item  Joshua Blumenstock et al. (2010).  A Method for Estimating the Relationship Between Phone Use and Wealth\cite{blumenstock2010method}  
\item  V. Frias-Martinez, Jesus Virseda (2012).  On the relationship between socio-economic factors and cell phone usage \cite{Frias_Martinez_2012} 
\item A. Llorente et al. (2014).  Social Media Fingerprints of Unemployment\cite{Llorente_2015}  
\item T. Gutierrez et al. (2013).  Evaluating socio-economic state of a country analyzing airtime credit and mobile phone datasets\cite{gutierrez2013evaluating}  
\item N. Eagle et al. (2010).  Network Diversity and Economic Development\cite{eagle2010network} Development  
\item Dong et al. (2014).  Inferring User Demographics and Social Strategies in Mobile Social Networks \cite{Dong:2014:IUD:2623330.2623703} 
\item Wang et al. (2015).  Forecasting Elections with Non-Representative Polls\cite{Wang2015980} 
\item C. Smith-Clarke et al. (2014).  Poverty on the Cheap: Estimating Poverty Maps Using Aggregated Mobile Communication Networks\cite{Smith-Clarke:2014:PCE:2556288.2557358} 
\end{enumerate}  
\\  \subsection{Migration, Mobility and Epidemiology using Big Data}  \begin{enumerate} 
\item Gonzales et al. (2008)  Understanding individual human mobility patterns\cite{Gonz_lez_2008} 
\item Wesolowski et al. (2013).  The impact of biases in mobile phone ownership on estimates of human mobility\cite{Wesolowski_2013}  
\item  Blumenstock, JE.(2012).  Inferring Patterns of Internal Migration from Mobile Phone Call Records: Evidence from Rwanda\cite{Blumenstock_2012}. Rwanda.  
\item  State, B. et al.(2014).  Migration of Professionals to the U.S.: Evidence from LinkedIn Data10.1007/978-3-319-13734-6_37\cite{State_2014}  
\item  Zagheni et al.(2014).  Inferring International and Internal Migration Patterns from Twitter Data\cite{Zagheni:2014:III:2567948.2576930}  
\item  Wesolowski, et al.(2012).  Quantifying the Impact of Human Mobility on Malaria \cite{Wesolowski_2012} 
\item Balcan et al. (2009).  Multiscale mobility networks and the spatial spreading of infectious diseases\cite{Balcan_2009} diseases  
\item Ginsberg et al. (2008).  Detecting Influenza Epidemics using Search Engine Query Data\cite{Ginsberg_2008} Data  
\item Pervaiz et al. (2012).  FluBreaks: Early Epidemic Detection from Google Flu Trends\cite{Pervaiz_2012} Trends  \item Eagle, Pentland (2005).  Reality mining: sensing complex social systems\cite{Eagle_2005} systems  
\item Onnela et al. (2007).  Structure and tie strengths in mobile communication networks\cite{Onnela_2007} networks  
\item Onnela et al. (2014).  Using sociometers to quantify social interaction patterns\cite{Onnela_2014} patterns  
\item Ratti et al. (2010).  Redrawing the map of Great Britain from a network of human interactions\cite{Ratti_2010} interactions  \item Amini et al. (2014).  The Impact of Social Segregation on Human Mobility in Developing and Urbanized Regions\cite{Amini_2014} Regions  
\item Onnela et al. (2011).  Geographic constraints on social network groups\cite{Onnela_2011} groups  
\item Cattuto et al. (2010).  Dynamics of person-to-person interactions from distributed RFID sensor networks\cite{Cattuto_2010} networks  
\item Muchnik et al. (2013).  Social Influence Bias: A Randomized Experiment\cite{Muchnik_2013}  
\item Aral et al. (2009).  Distinguishing influence-based contagion from homophily-driven diffusion in dynamic networks\cite{Aral_2009} networks  
\item M. Gomez-Rodriguez et al. (2012).  Inferring Networks of Diffusion and Influence\cite{Gomez_Rodriguez_2012} Influence  
\item  M. Gomez-Rodriguez et al. (2013). Modeling “Modeling  Information Propagation with Survival Theory\cite{rodriguez2013modeling} Theory  
\item J.E. Blumenstock , N. Eagle (2011).  Divided We Call: Disparities in Access and Use of Mobile Phones in Rwanda\cite{blumenstock2012divided} Rwanda  
\item  X. Lu et al. (2012). Predictability “Predictability  of Population Displacement after the 2010 Haiti Earthquake\cite{Lu_2012} Earthquake  
\item  Bengtsson et al. (2011) Improved "Improved  Response to Disasters and Outbreaks by Tracking Population Movements with Mobile Phone Network Data: A Post-Earthquake Geospatial Study in Haiti\cite{Bengtsson_2011}  
\item  Gao et al. (2014). Quantifying “Quantifying  Information Flow during Emergencies\cite{Gao_2014} Emergencies  
\item  Bagrow et al. 2011). Collective “Collective  Response of Human Populations to Large-Scale Emergencies\cite{Bagrow_2011}  \item  Wang et al. (2014). .  Learning to Detect Patterns of Crime\cite{Wang_2013} Crime  
\item  “Precinct or Prejudice? Understanding Racial Disparities in New York City’s Stop-and-Frisk Policy  \end{enumerate}  \\  \subsection{Data Science, Human Behavior and Networks}  \begin{enumerate}  
\item  Quang Duong Jessica Su  et al. (2013) . The Effect of Recommendations on Network Structure  
\item  Quang Duong.  Sharding Social Networks\cite{Duong_2013} Networks  
\item  Goel, Daniel  Goldstein.(2014)  Predicting Individual Behavior with Social Networks \cite{Goel_2014} 
\item  Daniel Reeves. Predicting without Markets  

\end{enumerate}  
\\  
\subsection{Information Flow and Consumption}  
\begin{enumerate}  
\item  Goel et Seth Flaxman. Filter Bubbles, Chambers and News Conumptions  
\item  Ashton Anderson et.  al.(2015).  The Structual Virality of Online Diffusion\cite{Goel_2015}  
\item  Dafna Shahaf, Carlos Guestrin (2012). Connecting the Dots between news articles Shahaf  . \cite{Shahaf:2012:CTD:2086737.2086744} \end{enumerate}  
\\  
\subsection{Advertisements and Recommendations}  
\begin{enumerate}  
\item Hill et al. (2006).  Network-Based Marketing: Identifying Likely Adopters via Consumer Networks\cite{Hill_2006}  
\item  Bhagat et al. (2012). Network-Based Marketing: Identifying Likely Adopters via Consumer Networks  
\item   Maximizing Product Adoption in Social Networks\cite{Bhagat:2012:MPA:2124295.2124368} 
\end{enumerate} 