\textit{Version 1. 10/25/2015}   
The focus of my research work up till now has been on Computational Social Science which can be considered as an intersection of Social Sciences and Computer Science. Computational Social Science is a big field employing different algorithmic tools and techniques to solve different problems in different areas of social sciences. My focus within Computational Social Science is on Human Behavior Analysis using Data Science Techniques. Study in this area will enable me to see that how different aspects of Human Behavior can be analyzed better using computational and statistical techniques.\\ 
Application of big data tools and techniques is becoming more and more important every day as the magnitude of information and data available for analysis has grown tremendously in the last decade. As a result, many research projects have been carried out which employ new and efficient techniques for mining human data. My research is focused on application of these big data tools and techniques on social problems. Some of the problems that I have been working on include Customer Churn Prediction using Telecommunication CDR data, Migration Trends Analysis using Telecommunication CDR Data and Latent Profile Features Predication using Twitter Data.\\ 

Human Behavior Analysis using Data Science is a natural progression of my background in Computer Science and theme of the projects that I have been involved in as a PhD Student in the Information School during the last 2 years.   Reading list for my general exam will consist of research material involving major techniques in data science and the application of the these techniques on the problems related to Social Science.   
\section{Methods (Data Science (Machine Learning) Literature)}
\begin{enumerate}   
\item  Tomaso Poggio and Steve Smale (2003).  The Mathematics of Learning: Dealing with Data\cite{Poggio_2005} 
\item  Z. Ghahramani (2004).  Unsupervised Learning \cite{Ghahramani_2004}
\item  X. Wu et al. (2008). Top 10 algorithms in data mining \cite{2009}  
\item  Hastie et al. (2009). Elements of Statistical Learning \cite{StatisticalLearning_2009}  
\item  Esther Duflo et al. (2006). Using Randomization in Development Economics Research: A Toolkit \cite{Duflo}   
\item  Rajaraman et al. (2009). Mining of Massive Datasets. \cite{Rajaraman_2009}  
\item  Yaser S. AbuMostafa et al. (2012). Learning from Data \cite{Abu-Mostafa:2012:LD:2207825}  
\end{enumerate}  
