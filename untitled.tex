\textit{Version 1. 10/25/2015} 

The focus of my research work up till now has been on Computational Social Science which can be considered as an intersection of Social Sciences and Computer Science. Computational Social Science is a big field employing different algorithmic tools and techniques to solve different problems in different areas of social sciences. My focus within Computational Social Science is on Human Behavior Analysis using Data Science Techniques. Study in this area will enable me to see that how different aspects of Human Behavior can be analyzed better using computational and statistical techniques.

Human Behavior Analysis using Data Science is a natural progression of my background in Computer Science and theme of the projects that I have been involved in as a PhD Student in the Information School during the last 2 years. Reading list for my general exam will consist of research material involving major techniques in data science and the application of the these techniques on the problems related to Social Science. 

\section{Data Science (Machine Learning) Literature}
\begin{itemize}
\item The Mathematics of Learning: Dealing with Data\cite{Poggio_2005}
\end{itemize}
\begin{itemize}
\item Unsupervised Learning \cite{Ghahramani_2004}
\end{itemize}
\begin{itemize}
\item Elements of Statistical Learning\cite{2009}
\end{itemize}
\begin{itemize}
\item Top 10 algorithms in data mining \cite{2009}
\end{itemize}
\begin{itemize}
\item Using Randomization in Development Economics Research: A Toolkit \cite{Duflo} 
\end{itemize}
